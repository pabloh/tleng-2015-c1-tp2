\documentclass[a4paper,8pt]{article}
\usepackage[utf8x]{inputenc}
\usepackage[margin=0.7in]{geometry}
% \usepackage[top=2in, bottom=1.5in, left=1in, right=1in]{geometry} -->
\usepackage{caratula}
\usepackage{graphicx}
\usepackage{listings}
\usepackage{algorithm}
\usepackage{algorithmic}
\usepackage{verbatim}
\usepackage{subfig}
\usepackage{enumerate}
\usepackage{hyperref}
\usepackage[table,xcdraw]{xcolor}
\usepackage{booktabs}

%uso: \ponerGrafico{file}{caption}{scale}{label}
\newcommand{\ponerGrafico}[4]
{\begin{figure}[H]
	\centering
	\subfloat{\includegraphics[scale=#3]{#1}}
	\caption{#2} \label{fig:#4}
\end{figure}
}

%%%%%%%%%%%%%%%%%%%%%%%%%%%%%%%%%%%%%%%%%%%%

\materia{Teor\'ia de Lenguajes}

\titulo{TP2}
\fecha{8/7/2015}
\grupo{PLD}
\integrante{Pablo Herrero}{332/07}{pablodherrero@gmail.com}
\integrante{Diego Sueiro}{75/90}{dsueiro@gmail.com}
\integrante{Leandro Tozzi}{-}{leandro.tozzi@gmail.com}


\include{templates}


\begin{document}
\pagestyle{myheadings}
\maketitle
\thispagestyle{empty}
\tableofcontents

%\setcounter{section}{-1}

\newpage

\section{Introducción}
En este trabajo desarrollamos un parser para el lenguaje musileng, creado por el cuerpo docente de la materia Teoría de Lenguajes de nuestra facultad. Este lenguaje esta orientado a la composición de piezas musicales, que luego será transformado al formato MIDI para su reproducción.

\section{Especificación de la gramática}
A continuación detallamos primero los tokens reconocidos por el lexer y luego las producciones reconocidas por el parser.
\subsection{Tokens}
En la siguiente tabla podemos observar los tokens reconocidos por el Lexer. Las cadenas representadas por cada token se corresponden con su nombre en minúscula salvo especificado lo contrario.

\vspace{1em}
\begin{tabular}{ll}
\hline
\rowcolor[HTML]{BBDAFF}
Token & Descripción\\
\hline
\texttt{HASH}				                      & Declaración de regla (carácter ``\#'')\\
\texttt{EQUALS}                               	  & Declaración de regla (carácter ``='')\\
\texttt{SLASH}  								  & Declaración de regla (carácter ``/'')\\
\texttt{ID}                                       & Constante alfanumérica (expresión regular \texttt{[\_a-zA-Z][\_a-zA-Z0-9]*})\\
\texttt{MUSICAL\_NOTE}                            & Constante alfanumérica que representa las notas musicales: \texttt{do re mi fa sol la si}\\
\hline
\texttt{NOTE\_VALUE}							  &\begin{tabular}[c]{@{}l@{}}Constantes que representan los valores de las notas musicales:\\
													 \texttt{``redonda'', ``blanca'', ``negra'', ``corchea'', ``semicorchea'',}\\
													 \texttt{``fusa'', ``semifusa''}
												   \end{tabular}\\
\hline
\texttt{SEMICOLON}          					  & Declaración de regla (carácter ``;'')\\
\texttt{COMMA}                                    & Declaración de regla (carácter ``,'')\\
\texttt{POINT}                                    & Declaración de regla (carácter ``.'')\\
\texttt{NUMBER}                                   & Constante numérica (expresión regular \texttt{[0-9]+(\textbackslash.[0-9]+)?})\\
\texttt{PLUS}, \texttt{MINUS}                     & Operadores aritméticos binarios (caracteres ``+'', ``-'')\\
\texttt{LPAREN}, \texttt{RPAREN}                  & Agrupación (caracteres ``('' y ``)'')\\
\texttt{LBRACKET}, \texttt{RBRACKET}              & Delimitación de bloques ``\{'' y ``\}'')\\
\texttt{COMMENT}                                  & Comentario (expresión regular \texttt{"([\^{}\textbackslash\textbackslash\textbackslash{}n]|(\textbackslash\textbackslash.))*"})\\
\hline
\begin{tabular}[c]{@{}l@{}} \texttt{TEMPO}, \texttt{COMPAS}, \texttt{CONST} \\
\texttt{VOZ}, \texttt{REPETIR}, \texttt{NOTA} \\ \end{tabular}\\  \texttt{SILENCIO}      & Keywords necesarias para el lenguaje\\
\hline
\end{tabular}

\subsection{Producciones}
Las producciones que definen la gramática se detallan a continuación, en el formato en el que son requeridas por la librería PLY. \\La producción inicial de esta gramática es \texttt{musileng}.

\begin{verbatim}
  musileng         : tempo_directive compas_directive constants voices

  tempo_directive  : HASH TEMPO NOTE_VALUE NUMBER
  
  compas_directive : HASH COMPAS NUMBER SLASH NUMBER
  
  constants        : constant constants
                   | 
                   
  constant         : CONST ID EQUALS NUMBER SEMICOLON
  
  voices           : voice voices
                   | voice
                   
  voice            : VOZ LPAREN numeric_value RPAREN LBRACKET voice_content RBRACKET
  
  
  
  
  voice_content    : compas voice_content
                   | repetition voice_content
                   | compas
                   | repetition
                   
  compas           : COMPAS LBRACKET compas_content RBRACKET
  
  compas_content   : note compas_content
                   | silence compas_content
                   | note
                   | silence
                   
  note             : NOTA LPAREN pitch COMMA numeric_value COMMA duration RPAREN SEMICOLON
  
  silence          : SILENCIO LPAREN duration RPAREN SEMICOLON
  
  repetition       : REPETIR LPAREN numeric_value RPAREN LBRACKET voice_content RBRACKET
  
  numeric_value    : NUMBER
                   | ID
                   
  duration         : NOTE_VALUE
                   | NOTE_VALUE POINT
                   
  pitch            : MUSICAL_NOTE
                   | MUSICAL_NOTE PLUS
                   | MUSICAL_NOTE MINUS
\end{verbatim}

\section{Implementación.}
\subsection{Parsing}

\section{Requerimientos}
	\begin{itemize}
  		\item Python 3.3 o superior
  		\item Librería PLY 3.6 (\url{http://www.dabeaz.com/ply/})
  		\item Midicomp (\url{https://github.com/markc/midicomp/})
	\end{itemize}

\section{Modo de uso}
Para hacer uso del programa simplemente hay que ejecutar el archivo musileng en directorio raiz pasándole en el primer parámetro el archivo de entrada y el nombre del archivo que se desea generar en el segundo. \\
Ejemplo: \texttt{./musileng ejemplos/ej1\_input.txt ejemplos/ej1\_output.txt }
\subsection{Tests}
Para correr la test suite hay que ejecutar dentro del directorio ./src, donde se encuentra el código fuente, el comando  \texttt{python -m unittest discover tests '*\_test.py'}


\section{Ejemplos}
ej
\subsection{Entrada inválida 1}
\subsection{Entrada inválida 2}
\newpage
\section{Decisiones de Diseño}
A continuación se enumeran las distintas decisiones de diseño que se tomaron en el transcurso del desarrollo del parser.
\begin{itemize}
\item Sobre el control de errores en el archivo de entrada se reporta la falla del primer error encontrado
\item Interpretamos que redefinir una constante es una falla del archivo de entrada, por lo tanto realizar este tipo de acción genera un mensaje de error.
\item La estructura \texttt{repetir(veces){}} puede aceptar tanto un número literal, como una constante
\item Los errores semánticos incluyen número de línea para facilitar la corrección por parte del programador/compositor de musileng
\item El número de track 10 debe ser asignado a percusión según la norma MIDI
\end{itemize}

\section{Conclusiones}
conclusiones

\newpage
\section{Código Fuente}

\subsection{parser.py}
\begin{small}
  \verbatiminput{../src/musileng/parser.py}
\end{small}
\newpage
\subsection{lexer.py}
\begin{small}
  \verbatiminput{../src/musileng/lexer.py}
\end{small}
\newpage
\subsection{ast.py}
\begin{small}
  \verbatiminput{../src/musileng/ast.py}
\end{small}
\newpage
\subsection{encoder.py}
\begin{small}
  \verbatiminput{../src/musileng/encoder.py}
\end{small}
\newpage
\subsection{visitor.py}
\begin{small}
  \verbatiminput{../src/musileng/visitor.py}
\end{small}
\newpage
\subsection{semantic\_analysis.py}
\begin{small}
  \verbatiminput{../src/musileng/semantic_analysis.py}
\end{small}
\newpage
\subsection{cli.py}
\begin{small}
  \verbatiminput{../src/musileng/cli.py}
\end{small}
\newpage

\end{document}
