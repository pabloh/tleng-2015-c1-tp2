\documentclass[a4paper,8pt]{article}
\usepackage[utf8x]{inputenc}
\usepackage[margin=0.7in]{geometry}
% \usepackage[top=2in, bottom=1.5in, left=1in, right=1in]{geometry} -->
\usepackage{caratula}
\usepackage{graphicx}
\usepackage{listings}
\usepackage{algorithm}
\usepackage{algorithmic}
\usepackage{subfig}
\usepackage{enumerate}
\usepackage{hyperref}
\usepackage[table,xcdraw]{xcolor}
\usepackage{booktabs}

%uso: \ponerGrafico{file}{caption}{scale}{label}
\newcommand{\ponerGrafico}[4]
{\begin{figure}[H]
	\centering
	\subfloat{\includegraphics[scale=#3]{#1}}
	\caption{#2} \label{fig:#4}
\end{figure}
}

%%%%%%%%%%%%%%%%%%%%%%%%%%%%%%%%%%%%%%%%%%%%

\materia{Teor\'ia de Lenguajes}

\titulo{TP2}
\fecha{8/7/2015}
\grupo{PLD}
\integrante{Pablo Herrero}{332/07}{pablodherrero@gmail.com}
\integrante{Diego Sueiro}{75/90}{dsueiro@gmail.com}
\integrante{Leandro Tozzi}{-}{leandro.tozzi@gmail.com}


\include{templates}


\begin{document}
\pagestyle{myheadings}
\maketitle
\thispagestyle{empty}
\tableofcontents

%\setcounter{section}{-1}

\newpage

\section{Introducción}
Intro

\section{Especificación de la gramática}
A continuación detallamos primero los tokens reconocidos por el lexer y luego las producciones reconocidas por el parser.
\subsection{Tokens}
En la siguiente tabla podemos observar los tokens reconocidos por el Lexer. Las cadenas representadas por cada token se corresponden con su nombre en minúscula salvo especificado lo contrario.

\vspace{1em}
\begin{tabular}{ll}
\hline
\rowcolor[HTML]{BBDAFF}
Token & Descripción\\
\hline
\texttt{HASH}				                      & Declaración de regla (carácter ``\#'')\\
\texttt{EQUALS}                               	  & Declaración de regla (carácter ``='')\\
\texttt{SLASH}  								  & Declaración de regla (carácter ``/'')\\
\texttt{ID}                                       & Constante alfanumérica (expresión regular \texttt{[\_a-zA-Z][\_a-zA-Z0-9]*})\\
\texttt{MUSICAL\_NOTE}                            & Constante alfanumérica que representa las notas musicales: \texttt{do re mi fa sol la si}\\
\hline
\texttt{NOTE\_VALUE}							  &\begin{tabular}[c]{@{}l@{}}Constantes que representan los valores de las notas musicales:\\
													 \texttt{``redonda'', ``blanca'', ``negra'', ``corchea'', ``semicorchea'',}\\
													 \texttt{``fusa'', ``semifusa''}
												   \end{tabular}\\
\hline
\texttt{SEMICOLON}          					  & Declaración de regla (carácter ``;'')\\
\texttt{COMMA}                                    & Declaración de regla (carácter ``,'')\\
\texttt{POINT}                                    & Declaración de regla (carácter ``.'')\\
\texttt{NUMBER}                                   & Constante numérica (expresión regular \texttt{[0-9]+(\textbackslash.[0-9]+)?})\\
\texttt{PLUS}, \texttt{MINUS}                     & Operadores aritméticos binarios (caracteres ``+'', ``-'')\\
\texttt{LPAREN}, \texttt{RPAREN}                  & Agrupación (caracteres ``('' y ``)'')\\
\texttt{LBRACKET}, \texttt{RBRACKET}              & Agrupación (caracteres ``('' y ``)'')\\
\texttt{COMMENT}                                  & Comentario (expresión regular \texttt{"([\^{}\textbackslash\textbackslash\textbackslash{}n]|(\textbackslash\textbackslash.))*"})\\
\hline
\begin{tabular}[c]{@{}l@{}} \texttt{TEMPO}, \texttt{COMPAS}, \texttt{CONST} \\
\texttt{VOZ}, \texttt{REPETIR}, \texttt{NOTA} \\ \end{tabular}\\  \texttt{SILENCIO}      & Keywords necesarias para el lenguaje\\
\hline
\end{tabular}

\subsection{Producciones}

\section{Implementación.}
\subsection{Parsing}

\section{Requerimientos}
	\begin{itemize}
  		\item Python 3.3 o superior
  		\item Librería PLY 3.6 (\url{http://www.dabeaz.com/ply/})
  		\item Librería Midi-Completar (\url{https://www.midi/})
	\end{itemize}

\section{Modo de uso}
modo de uso


\section{Ejemplos}
ej
\subsection{Entrada inválida 1}
\subsection{Entrada inválida 2}



\section{Conclusiones}
conclusiones

\newpage

\end{document}
