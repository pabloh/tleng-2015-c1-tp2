\documentclass[a4paper,8pt]{article}
\usepackage[utf8x]{inputenc}
\usepackage[margin=0.7in]{geometry}
% \usepackage[top=2in, bottom=1.5in, left=1in, right=1in]{geometry} -->
\usepackage{caratula}
\usepackage{graphicx}
\usepackage{listings}
\usepackage{algorithm}
\usepackage{algorithmic}
\usepackage{subfig}

%uso: \ponerGrafico{file}{caption}{scale}{label}
\newcommand{\ponerGrafico}[4]
{\begin{figure}[H]
	\centering
	\subfloat{\includegraphics[scale=#3]{#1}}
	\caption{#2} \label{fig:#4}
\end{figure}
}

%%%%%%%%%%%%%%%%%%%%%%%%%%%%%%%%%%%%%%%%%%%%

\materia{Teor\'ia de Lenguajes}

\titulo{TP1}
\fecha{29/4/2015}
\grupo{PLD}
\integrante{Pablo Herrero}{332/07}{pablodherrero@gmail.com}
\integrante{Diego Sueiro}{75/90}{dsueiro@gmail.com}
\integrante{Leandro Tozzi}{-}{leandro.tozzi@gmail.com}


\include{templates}


\begin{document}
\pagestyle{myheadings}
\maketitle
\thispagestyle{empty}
\tableofcontents

%\setcounter{section}{-1}

\newpage

\begin{section}{Introducción}
El Trabajo Práctico consiste en implementar un programa que permita la construcción y ejecución de Autómatas Finitos Determinísticos (AFD's).

\end{section}

\begin{section}{Implementación.}

La implementación se realizó en Python v3 utilizando el esqueleto provisto por la cátedra. Se proveen los test de unidad y los archivos de entrada de datos utilizados para verificar el correcto funcionamiento del programa.

\end{section}
\begin{subsection}{Representación Autómata}
Se utiliza la clase \textbf{Automata} para representar la estructura y los algoritmos necesarios para procesar los autómatas.
Mediante conjuntos almacenamos los estados totales, los estados finales y el lenguaje a utilizar.
La tabla de transiciones se representa mediante un diccionario.

\end{subsection}

\begin{subsection}{Lectura Expresion Regular}
Mediante la función \textbf{ReadFromFile} se procesa el archivo de entrada que contiene la expresión regular parseada. Se chequea que solo contenga las operaciones y los caracteres aceptados por el enunciado, así como también, el indentado de las mismas. La función \textbf{ReadFromFile} devuelve una lista que contiene las operaciones de la ER a realizar, así como también el numero de operandos a utilizar por cada operación
\end{subsection}

\begin{subsection}{Conversion AFND a AFD}
El primer punto del TP plantea leer una ER y generar un AFD mínimo.
Una vez procesado el archivo de entrada que contiene la expresión regular, se genera un AFND-$\lambda$ mediante la clase \textbf{AFNDfromER}. Esta clase utiliza como soporte la clase \textbf{BuilderAF} que recibe simplemente agrega los estados y las transiciones $\lambda$ necesarias para construir el autómata que representa la ER que le pasamos.
En base a este AFND resultante, se construye un AFD  mediante la clase \textbf{AFDfromAFN}

\ponerGrafico{AFND_1.png}{Conversion AFND a AFD: Autómata de Entrada AFND}{0.4}{}
\ponerGrafico{AFND_2.png}{Conversion AFND a AFD: Autómata resultante AFD}{0.6}{}
\end{subsection}

\begin{subsection}{Lectura de AFD desde archivo de entrada}
Se procesa el archivo de entrada y se genera un AFD mediante la clase \textbf{AFDfromFile}. No se implementó chequeo de errores. Asuminos que el archivo de entrada cumple con la estructura planteada en el enunciado.
\end{subsection}

\begin{subsection}{Generación de archivo .dot}
Se generan correctamente los archivos .dot para ser visualizados mediante GraphViz. Adicionalmente se genera el archivo gráfico en formato png.
\end{subsection}

\begin{subsection}{Minimización}
Se implementó el algoritmo de minimización por particionado en clases equivalentes. Se comienza desde un particionado inicial de 2 grupos, estados finales y estados no finales.
Luego se recorre cada estado y se chequea a que grupo iría mediante cada una de los símbolos del lenguaje.
De esta manera vamos agrupando en clases equivalentes a los estados. 
El algoritmo termina cuando no se puede volver a particionar. 
Luego eliminamos los estados inalcanzables del resultado para poder rearmar el autómata M mínimo

%PseudoCodigo del algoritmo de minimizacion
\begin{algorithm}[H]
\begin{algorithmic}[1]
\REQUIRE M automata finito deterministico
\ENSURE M minimo
\STATE grupoA $:=$ \{Estados Finales\}
\STATE grupoB $:=$ \{Estados no Finales\}
\REPEAT 
	\FORALL {grupo}
		\FORALL {estado}
			\STATE Encontrar a que grupo nos llevan las entradas
			\IF {Hay diferencias}
				\STATE Particionar el grupo en conjuntos cuyos estados\\ 
						vayan a los mismos grupos bajo esa entrada
			\ENDIF
		\ENDFOR
	\ENDFOR
\UNTIL {(no haya nuevo particionado)}
\STATE RemoverEstadosInalcanzables(M)
\STATE M $\leftarrow$ grupos como estados
\caption{Minimizar(automata M)}
\end{algorithmic}
\end{algorithm}

\ponerGrafico{min_input.png}{Ejemplo de Minimización: Autómata de Entrada - sin minimizar}{0.6}{}
\ponerGrafico{min_final.png}{Ejemplo de Minimización: Autómata minimizado}{0.6}{}

\end{subsection}
\newpage
\begin{subsection}{Intersección de Autómatas}
Se implementó la operación \textbf{Cross Product} entre autómatas para la realización de este item. 

\ponerGrafico{intersectar.png}{Ejemplo de Intersección: Autómatas a intersectar}{0.6}{}
\ponerGrafico{intersectar-res.png}{Ejemplo de Intersección: Resultado}{0.6}{}

\end{subsection}

\begin{subsection}{Complemento}
Primero completamos el autómata mediante el método \textbf{complete\_all} de la clase \textbf{autómata}.
Luego se llamá al método \textbf{complemento} de la misma clase para realizar el complemento de su lenguaje del autómata. 
\end{subsection}

\begin{subsection}{Equivalencia de Autómatas}
En este punto se aprovecharon las funciones implementadas previamente, ya que calculamos que si la intersección del complemento es \textbf{NULL} en ambas direcciones, entonces los lenguajes generados por los autómatas son equivalentes
\end{subsection}

\begin{section}{Conclusiones}
La realización del trabajo práctico nos permitió famiarizarnos con los algoritmos vistos en la teórica para el manejo de expresiones regulares y autómatas.
\end{section}
\newpage

\end{document}
