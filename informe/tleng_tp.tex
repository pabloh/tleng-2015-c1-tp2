\documentclass[a4paper,8pt]{article}
\usepackage[utf8x]{inputenc}
\usepackage[margin=0.7in]{geometry}
% \usepackage[top=2in, bottom=1.5in, left=1in, right=1in]{geometry} -->
\usepackage{caratula}
\usepackage{graphicx}
\usepackage{listings}
\usepackage{algorithm}
\usepackage{algorithmic}
\usepackage{subfig}
\usepackage{enumerate}
\usepackage{hyperref}

%uso: \ponerGrafico{file}{caption}{scale}{label}
\newcommand{\ponerGrafico}[4]
{\begin{figure}[H]
	\centering
	\subfloat{\includegraphics[scale=#3]{#1}}
	\caption{#2} \label{fig:#4}
\end{figure}
}

%%%%%%%%%%%%%%%%%%%%%%%%%%%%%%%%%%%%%%%%%%%%

\materia{Teor\'ia de Lenguajes}

\titulo{TP2}
\fecha{8/7/2015}
\grupo{PLD}
\integrante{Pablo Herrero}{332/07}{pablodherrero@gmail.com}
\integrante{Diego Sueiro}{75/90}{dsueiro@gmail.com}
\integrante{Leandro Tozzi}{-}{leandro.tozzi@gmail.com}


\include{templates}


\begin{document}
\pagestyle{myheadings}
\maketitle
\thispagestyle{empty}
\tableofcontents

%\setcounter{section}{-1}

\newpage

\section{Introducción}
Intro


\section{Especificación de la gramática}
espec
\subsection{Tokens}
Tabla tokens
\subsection{Producciones}

\section{Implementación.}
\subsection{Parsing}

\section{Requerimientos}
	\begin{itemize}
  		\item Python 3.3 o superior
  		\item Librería PLY 3.6 (\url{http://www.dabeaz.com/ply/})
  		\item Librería Midi-Completar (\url{https://www.midi/})
	\end{itemize}

\section{Modo de uso}
modo de uso


\section{Ejemplos}
ej
\subsection{Entrada inválida 1}
\subsection{Entrada inválida 2}



\section{Conclusiones}
conclusiones

\newpage

\end{document}
